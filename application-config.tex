%% -----------------------------------------------------
%% Configuration of an NSERC application
%% Defines a few macros that are global to all documents of the application
%% ----------------------------------------------------

%% If your application involves a company, put the name of the company
%% in a macro rather than writing it directly.
\newcommand{\namecompany}{Denso Corporation}

%% Application year. Only appears in the metadata of the generated PDF.
\newcommand{\applicationyear}{2025}

%% The list of all authors of the application. Again, only useful for the
%% PDF metadata
\newcommand{\authorlist}{Emmett Brown}

%% The name and NSERC PIN of the main applicant
\nsercname{Liam Paull}
\nsercpin{263120}

%% Documents are not dated
\date{}

%% Paragraphes français
%\setlength{\parindent}{0pt}

%% Hack to have list items displayed in a more compact way
\usepackage{paralist}
\setlength{\pltopsep}{4pt}
\setlength{\plitemsep}{4pt}

%% ----------
%% Loading a few packages. These are all optional and can be commented
%% out if you with. Feel free to add others.
%% ----------
\usepackage{hyperref}
\hypersetup{%
  pdfauthor = {\authorlist{}},
  pdfcreator = {NSERC Alliance LaTeX Template V1.1},
  pdfsubject = {NSERC \applicationyear{} \namecompany{}}
}
\usepackage{url}
\usepackage{todonotes}
\usepackage{graphicx}
\usepackage{enumitem}

%% ------------------------
%% Useful: a few "todo" macros to display colored boxes with remarks
%% and comments
%% ------------------------
%%\newcommand{\todo}[1]{\todo[inline,caption={},color=yellow]{\sf\small #1}}

%% ------------------------
%% Color for grayed out instruction bullets in the text. Change
%% this definition to show instructions with a different shade.
%% Comment it out to revert the instructions to black like the rest
%% of the text.
%% ------------------------
\definecolor{instructioncl}{gray}{0.35}

%% ------------------------
%% This will print a "DRAFT" watermark on all pages.
%% Uncomment the next two lines once the application is ready.
%% ------------------------
% \usepackage{draftwatermark}
% \SetWatermarkText{DRAFT}
